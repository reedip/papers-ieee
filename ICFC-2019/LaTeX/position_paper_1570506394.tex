\documentclass[conference]{IEEEtran}
\IEEEoverridecommandlockouts
% The preceding line is only needed to identify funding in the first footnote. If that is unneeded, please comment it out.
\usepackage{cite}
\usepackage{amsmath,amssymb,amsfonts}
\usepackage{algorithmic}
\usepackage{graphicx}
\usepackage{textcomp}
\usepackage{xcolor}
\usepackage{url}
\def\BibTeX{{\rm B\kern-.05em{\sc i\kern-.025em b}\kern-.08em
    T\kern-.1667em\lower.7ex\hbox{E}\kern-.125emX}}
\begin{document}

\title{Edge Clouds Control Plane and Management Data Consistency
Challenges: Position Paper for IEEE International Conference on Cloud
Engineering, 2019\\ }

\author{\IEEEauthorblockN{Bohdan Dobrelia}
\IEEEauthorblockA{\textit{OpenStack platform} \\
\textit{Red Hat}\\
Poznan, Poland \\
bdobreli@redhat.com}
}

\maketitle

\begin{abstract}
Fog computing is emerging Cloud of (Edge) Clouds technology. Its control plane
and deployments data synchronization is a major challenge. Autonomy requirements
expect even the most distant edge sites always manageable, available for
monitoring and alerting, scaling up/down, upgrading and applying security fixes.
Whenever temporary disconnected sites are managed locally or centrally, some
changes and data need to be eventually synchronized back to the central site(s)
with having its merge-conflicts resolved for the central data hub(s). While
some data needs to be pushed from the central site(s) to the Edge, which might
require resolving data collisions at the remote sites as well. In this paper,
we position the outstanding data synchronization problems for OpenStack
platform becoming a cloud solution number one for fog computing. We define the
inter-cloud operational invariants based on that Always Available autonomy
requirement. We show that a causally consistent key value storage is the best
match for the outlined operational invariants and there is a great opportunity
for designing such a solution for Edge clouds. Finally, the paper brings the
vision of unified tooling to solve the data synchronization problems the same
way for infrastructure owners, IaaS cloud operators and tenants running
workloads for PaaS, like OpenShift or Kubernetes deployed on top of Edge
clouds.
\end{abstract}

\begin{IEEEkeywords}
Open source software, Edge computing, Distributed computing, System
availability, Design
\end{IEEEkeywords}


\section{Glossary}

Aside of the established terms \cite{b3}, we define a few more for the data
processing and operational points of view:

\subsection{Deployment Data}

Data that represents the configuration of cloudlets \cite{b3}, like API
endpoints URI, or numbers of deployed edge nodes \cite{b3} in edge clouds
\cite{b3}. That data represents the most recent state of a deployment.

\subsection{Cloud Data}

Represents the most recent internal and publicly visible state of cloudlets,
like cloud users or virtual routers. Cloud data also includes logs, performance
and usage statistics, state of message queues and the contents of databases.
When we refer to just data, we do not differentiate either that is deployment
or control data. When there is unresolved data merging conflicts, the most
recent state becomes the best known state.

\subsection{Control Plane (CP)}

Corresponds to any operations performed via cloudlets API endpoints
or CLI tooling. For example, starting a virtual machine instance, or creating a
cloud user. Such operations are typically initiated by cloud applications,
tenants or operators.

\subsection{Management Plane}

Corresponds to administrative actions performed via configuration and lifecycle
management systems. Such operations are typically targeted for cloudlets, like
edge nodes, edge datacenters \cite{b3}, or edge clouds. For example, upgrading
or reconfiguring the centralized data center \cite{b3}, or scaling up edge
nodes. And typically initiated by cloud infrastructure owners. For some
extendned cases, like Baremetal-as-a-Service, tenants may as well initiate
actions executed via the management plane. Collecting logs, performance and
usage statistics for monitoring and alerting systems also represents the
management plane operations, although it operates with the cloud data.

\subsection{Always Available (AA)}

The operational mode of the control and management planes that corresponds to
the best for today choices for the sticky available consistency models
\cite{b4}, which is like Real-Time Causal \cite{b2}, or causal+ \cite{b1}.
The

\section{Introduction}

\section{Analysis and Discussion}

\subsection{Autonomy Requirements}

We define autonomy requirements for Edge sites as the following:

\begin{itemize}
\item Foo
\end{itemize}

\subsection{Operational Invariants}

Always available control and management planes require the following
operational capabilities:

\begin{itemize}
\item Foo
\end{itemize}

\subsection{Data Consistency Requirements}

Foo \cite{b2}.

\section{Conclusion}

\subsection{Figures and Tables}
\paragraph{Positioning Figures and Tables} Place figures and tables at the top and
bottom of columns. Avoid placing them in the middle of columns. Large
figures and tables may span across both columns. Figure captions should be
below the figures; table heads should appear above the tables. Insert
figures and tables after they are cited in the text. Use the abbreviation
``Fig.~\ref{fig}'', even at the beginning of a sentence.

\begin{table}[htbp]
\caption{Table Type Styles}
\begin{center}
\begin{tabular}{|c|c|c|c|}
\hline
\textbf{Table}&\multicolumn{3}{|c|}{\textbf{Table Column Head}} \\
\cline{2-4}
\textbf{Head} & \textbf{\textit{Table column subhead}}& \textbf{\textit{Subhead}}& \textbf{\textit{Subhead}} \\
\hline
copy& More table copy$^{\mathrm{a}}$& &  \\
\hline
\multicolumn{4}{l}{$^{\mathrm{a}}$Sample of a Table footnote.}
\end{tabular}
\label{tab1}
\end{center}
\end{table}

\begin{figure}[htbp]
\centerline{\includegraphics[scale=0.5]{fig1.png}}
\caption{Example of a figure caption.}
\label{fig}
\end{figure}

\begin{thebibliography}{00}
\bibitem{b1} W. Lloyd, M. J. Freedman, M. Kaminsky, and D. G. Andersen, ``Don’t settle for eventual: Scalable causal consistency for wide-area storage with COPS,'' Proc. 23rd ACM Symposium on Operating Systems Principles (SOSP 11), Cascais, Portugal, October 2011.
\bibitem{b2} P. Mahajan, L. Alvisi, and M. Dahlin. ``Consistency, availability, and convergence,'' Technical Report TR-11-22, Univ. Texas at Austin, Dept. Comp. Sci., 2011.
\bibitem{b3} The Linux Foundation, ``Open Glossary of Edge Computing,'' [Online]. Available: \url{https://github.com/State-of-the-Edge/glossary}
\bibitem{b4} K. Kingsbury, ``Consistency Models,'' [Online]. Available: \url{https://jepsen.io/consistency}
\end{thebibliography}
\end{document}
